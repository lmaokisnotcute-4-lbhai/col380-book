\chapter{Lecture 7: OpenMP Scheduling Strategies, Reduction, and Performance Optimisation}
\noindent\textit{Transcribed by Devanshi Malik (2022CS51142)}
\section{Work Sharing Constructs in OpenMP}

This lecture continued the discussion on OpenMP and focused on work sharing
constructs, particularly the \texttt{for} construct. In OpenMP, loop iterations
are divided among threads, and the scheduling strategy determines how this
division is performed.

The scheduling strategy can either be explicitly specified by the programmer
or left to the runtime system. In some cases, expert programmers may choose an
appropriate strategy manually, while in others the runtime system makes the
decision.

\subsection{Static Scheduling}

In static scheduling, iterations are divided among threads before execution
begins. The assignment is fixed and typically follows a round-robin or block
distribution. Each thread is assigned a chunk of iterations, where the chunk
size determines how many iterations are grouped together.

Static scheduling works well when:
\begin{itemize}
  \item The workload per iteration is uniform.
  \item The iteration space and workload are known in advance.
\end{itemize}

However, static scheduling does not adapt during execution and cannot respond
to load imbalance if different iterations take different amounts of time.

\subsection{Dynamic Scheduling}

In dynamic scheduling, iterations are assigned to threads on demand. A thread
requests a new chunk of work once it finishes its current chunk. This allows
better load balancing when iteration workloads are uneven.

Dynamic scheduling introduces additional runtime overhead due to coordination
and synchronization. However, it can significantly reduce idle time when the
workload per iteration is irregular, since threads only request new work when
they become idle.

\subsection{Guided Scheduling}

Guided scheduling is similar to dynamic scheduling, but the chunk size decreases
over time. Initially, large chunks are assigned to reduce scheduling overhead,
and as execution progresses, the chunk size becomes smaller. This strategy
attempts to balance the trade-off between scheduling overhead and load
balancing.
The following examples from the lecture slides illustrate the use of different
OpenMP scheduling strategies for loop iteration spaces.

\begin{ppexample}
The following examples illustrate how different OpenMP scheduling strategies
control the assignment of loop iterations to threads.
\end{ppexample}

\begin{ppprogram}{Static scheduling with fixed chunk size}{prog:omp-static}
\begin{lstlisting}[language=OpenMP]
#pragma omp parallel for schedule(static, chunk_size)
for (int i = 0; i < N; i++) {
  do_stuff();  // e.g., a[i] += b[i]
}
\end{lstlisting}
\end{ppprogram}

Static scheduling assigns iterations in advance, typically in a round-robin
manner, and works well when the workload per iteration is uniform.

\begin{ppprogram}{Dynamic scheduling with on-demand assignment}{prog:omp-dynamic}
\begin{lstlisting}[language=OpenMP]
#pragma omp parallel for schedule(dynamic, chunk_size)
for (int i = 0; i < N; i++) {
  do_stuff();  // e.g., a[i] += b[i]
}
\end{lstlisting}
\end{ppprogram}

Dynamic scheduling incurs higher runtime overhead but provides better load
balancing when iteration costs vary.


\section{Reduction Operations}

Reduction operations combine values produced by multiple threads into a single
result using an associative operator such as addition or multiplication.

Common reduction operators include:
\[
+, \times, \min, \max, \land, \lor
\]

Reduction operations help avoid data races that would otherwise occur when
multiple threads update a shared variable. Instead of manually using atomic
operations or critical sections, OpenMP provides built-in support for reductions.

Reduction operators are typically associative and commutative. Although multiple
reduction clauses may appear in a program, each reduction clause specifies a
single operator applied to a particular variable.
The following example demonstrates the use of a reduction clause to safely
accumulate values across threads.

\begin{ppexample}
OpenMP reductions provide a concise way to avoid data races when combining
values computed by multiple threads.
\end{ppexample}

\begin{ppprogram}{Reduction for summation}{prog:omp-reduction}
\begin{lstlisting}[language=OpenMP]
double sum = 0.0;

#pragma omp parallel for reduction(+:sum)
for (int i = 0; i < N; i++)
  sum += a[i];
\end{lstlisting}
\end{ppprogram}

\begin{pppitfall}
Reductions avoid races on the reduced variable, but they do not automatically
make all code correct. Shared data structures, I/O operations, and
non-associative floating-point computations may still require careful handling.
\end{pppitfall}



\section{Sections Construct and Task-Based Parallelism}

The \texttt{sections} construct enables task-based parallelism in OpenMP.
Different sections correspond to different tasks, and each section is executed
by exactly one thread.

Key properties of the \texttt{sections} construct include:
\begin{itemize}
  \item Each section is executed exactly once.
  \item There is an implicit barrier at the end of the \texttt{sections} region.
  \item The assignment of sections to threads is determined at runtime.
\end{itemize}

The order in which sections are executed is unspecified. Although sections may
execute in parallel, there is no guarantee that all sections will do so
simultaneously. If sections modify shared data, explicit synchronization is
required to ensure correctness.
The \texttt{sections} construct can be used to express task-based parallelism, as
shown in the following example from the lecture slides.

\begin{ppexample}
The \texttt{sections} construct enables task-based parallelism by assigning
different tasks to different threads.
\end{ppexample}

\begin{ppprogram}{Task parallelism using sections}{prog:omp-sections}
\begin{lstlisting}[language=OpenMP]
#pragma omp parallel
{
  #pragma omp sections
  {
    #pragma omp section
    { compute_A(); }

    #pragma omp section
    { compute_B(); }

    #pragma omp section
    { compute_C(); }
  } // implicit barrier at end of sections (unless nowait)
}
\end{lstlisting}
\end{ppprogram}


\section{Performance Optimisation Tips}

Thread creation and context management introduce overheads. If the iteration
space is small, the overhead of parallel execution may outweigh its benefits,
leading to slower execution.

False sharing can occur when multiple threads frequently access independent
variables that reside on the same cache line. Padding data structures can help
separate such variables across cache lines and reduce performance degradation.

Choosing an appropriate scheduling strategy is crucial. Programmers should
consider workload regularity, iteration dependencies, and chunk sizes when
designing parallel loops.

Synchronization should be minimized whenever possible. Some parallel regions
contain implicit barriers that may lead to unnecessary synchronization.
Carefully using clauses such as \texttt{nowait} can help reduce this overhead
when synchronization is not required.

The lecture also highlighted the use of the \texttt{nowait} clause to avoid
unnecessary synchronization barriers when it is safe to do so.

\begin{ppexample}
The \texttt{nowait} clause can be used to eliminate unnecessary synchronization
when it is safe to do so.
\end{ppexample}

\begin{ppprogram}{Avoiding unnecessary barriers with nowait}{prog:omp-nowait}
\begin{lstlisting}[language=OpenMP]
#pragma omp parallel
{
  #pragma omp for nowait
  for (int i = 0; i < n; i++) {
    process_part1(i);
  }

  #pragma omp for
  for (int i = 0; i < n; i++) {
    process_part2(i);
  }
}
\end{lstlisting}
\end{ppprogram}



If the iteration space size is sufficiently large (for example, on the order of
$10^{4}$ to $10^{5}$ iterations or more), parallelization is more likely to be
beneficial. For smaller iteration spaces, parallel overheads may dominate and
mask any performance gains.

Important design considerations include:
\begin{itemize}
  \item Workload regularity, which influences the choice between static,
        dynamic, and guided scheduling.
  \item Appropriate selection of chunk size based on the number of threads.
\end{itemize}

These design decisions often depend on the programmer’s experience and the
specific characteristics of the workload.

\section{SIMD Vectorisation}

Some computations can be accelerated using SIMD (Single Instruction, Multiple
Data) vector instructions, where a single instruction operates on multiple data
elements simultaneously. When applicable, vectorisation can significantly
improve performance.

Vectorisation is generally unsafe or ineffective when:
\begin{itemize}
  \item There are loop-carried dependencies.
  \item Memory accesses are poorly aligned.
  \item The loop contains complex control flow or branching.
\end{itemize}
The following example shows how the compiler can be explicitly guided to
vectorize a loop using OpenMP SIMD directives.

\begin{ppexample}
The compiler can be explicitly guided to vectorize loops using OpenMP SIMD
directives.
\end{ppexample}

\begin{ppprogram}{SIMD vectorisation of a loop}{prog:omp-simd}
\begin{lstlisting}[language=OpenMP]
#pragma omp parallel for simd
for (int i = 0; i < n; i++) {
  result[i] = a[i] + b[i] * c[i];
}
\end{lstlisting}
\end{ppprogram}

\begin{ppprogram}{Increasing parallelism using collapse}{prog:omp-collapse}
\begin{lstlisting}[language=OpenMP]
#pragma omp parallel for collapse(2)
for (int i = 0; i < 100; i++) {
  for (int j = 0; j < 100; j++) {
    matrix[i][j] = compute(i, j);
  }
}
\end{lstlisting}
\end{ppprogram}


Common loop optimisation techniques include:
\begin{itemize}
  \item \emph{Loop fusion}, which reduces loop overhead and improves
        instruction-level parallelism. The \texttt{collapse} clause can be used
        to combine nested loops into a single iteration space when appropriate.
  \item \emph{Loop unrolling}, which reduces loop-control overhead and can
        improve cache utilization by fetching data in bursts.
  \item Avoiding the use of locks inside loops whenever possible.
  \item Avoiding memory allocation inside parallel regions.
\end{itemize}

Determining the optimal unrolling factor often requires experimentation.
Programmers may profile performance to identify cases where threads spend
significant time idle while waiting for new work, and adjust unrolling
accordingly.
